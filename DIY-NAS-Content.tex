\section{Initial situation}

In 2010 I bought my first \gls{NAS}. It was a \gls{Synology}
\href{https://global.download.synology.com/download/Document/Hardware/DataSheet/DiskStation/11-year/DS411slim/ger/Synology_DS411slim_Data_Sheet_ger.pdf}{DS411slim}.
The device is running up to now. For a few years now, it is only supplied with
security updates. That is quite remarkable for the fact that it has already 12
years on the hump.
\bigbreak

Now I've run out of space - I have 4*2TB running there in \href{https://en.wikipedia.org/wiki/RAID}{RAID 10},
which results in about 3.7TB. In addition, the performance is, well, not quite up
to date.

\section{Search for alternatives}

\subsection{Use a mini PC in front of the NAS}

Realizing this situation I was searching for alternatives. The first idea was
to use a \href{https://www.zotac.com/at/product/mini_pcs/id41-plus}{mini PC} in
front of the \gls{NAS}. This failed for some reasons.

\begin{itemize}
    \item Using the \gls{NAS} directly for \gls{Docker} volumes
    \item The CPU of the mini PC does not support \href{https://en.wikipedia.org/wiki/X86_virtualization}{Intel VT}
    \item The network as bottleneck for \gls{Docker} containers
\end{itemize}

\subsection{Hardware alternatives}

Then I decided to by a new \gls{NAS} system. But which kind of \gls{NAS} will
it be? I've spent a long time to search the internet for possible soultions.
I didn't know about the possibility to pimp a \href{https://www.terra-master.com}{TerraMaster \gls{NAS}}
with a different \href{https://www.youtube.com/watch?v=f9Fh8XgREJ0}{OS}. This
is also possible for \href{https://www.westerndigital.com}{Western Digital \gls{NAS}}
as described \href{https://community.wd.com/t/firmware-freenas-on-pr4100-updated/218730/11}{here}
and \href{https://forum.openmediavault.org/index.php?thread/37009-can-i-install-omv-on-a-wd-nas/}{here}.
Also the variant with a mini PC and separated storage case was interesting. The
idea is to get the most out of money, so the ranking here is done by
price/performance ratio and the DIY \gls{NAS} wins the comparison.

\begin{itemize}
    \item DIY \gls{NAS} system on PC base
    \item DIY \gls{NAS} system with separated storage case and mini PC
    \item Commercial \gls{NAS} system
    \item Commercial \gls{NAS} with custom OS
\end{itemize}

\subsection{The planned hardware}

When comparing the different hardware solutions, I also made a comparison
between an Intel and an AMD based \gls{NAS}. The AMD variant won the price/power
chapter, so I decided to buy:

\begin{itemize}
    \item 32GB G.Skill RipJaws V black DDR4-3200 DIMM Dual Kit
    \item 250GB Samsung 970 Evo Plus M.2 2280
    \item 400 Watt be quiet! Pure Power 11 CM Modular 80+ Gold
    \item Black Fractal Design Node 304 cube without power supply
    \item 4x 4000GB WD Red Plus WD40EFZX 128MB 3.5"
    \item ASRock Fatal1ty B450 Gaming-ITX/AC AMD B450
    \item AMD Athlon 3000G with Radeon Vega Graphics 3.5GHz
    \item Noctua NH-L9a-AM4 topblow cooler
\end{itemize}

\subsection{The bought hardware}

\tsImage{./images/John-Lennon-Quote.jpg}{John Lennon}{width=\textwidth}

This means that the availability of chips and other events affect the market
and make it difficult to realize these plans. For example, motherboards in
mini-ITX format with AM4 socket are really hard to get. That's why I revised
my decision on the components:

\begin{itemize}
    \item 32GB G.Skill RipJaws V black DDR4-3200 DIMM Dual Kit
    \item 250GB Samsung 970 Evo Plus M.2 2280
    \item 400 Watt be quiet! Pure Power 11 CM Modular 80+ Gold
    \item Black Fractal Design Node 304 cube without power supply
    \item 4x 4000GB WD Red Plus WD40EFZX 128MB 3.5"
    \item ASRock H610M-ITX/AC mITX Intel H610 DDR4 S1700
    \item Intel Core I3-12100 tray
    \item Noctua NH-L9i-17xx topblow cooler
\end{itemize}

The difference is that the CPU is a latest generation
\href{https://www.intel.com/content/www/us/en/products/sku/134584/intel-core-i312100-processor-12m-cache-up-to-4-30-ghz/specifications.html?wapkw=12100}{Intel I3}
processor. The corresponding
\href{https://www.asrock.com/MB/Intel/H610M-ITXac/index.de.asp}{motherboard}
is also brand new. Both were only launched in Q4/2022. This has some
consequences: \gls{OMV} 5 does not support the network chip of the motherboard.
So I decided to install the \gls{OMV} 6 beta on my DIY \gls{NAS}. I also tried
\href{https://www.truenas.com/}{TrueNAS} -- it looks much more professional
than \gls{OMV}, but it is also a bit more complicated to understand and
configure. It also doesn't have native support for \gls{Docker}.

\section{The hardware build}

Although I build up my last PC more than 25 years ago assembling the
hardware was a breeze. The only surprise was the mounting position of the
power supply. As I understand how to do this everything was fine.

\section{The software installation}

\subsection{The operating system}

The basic installation of \gls{OMV} works without any problems. This is known
from debian. Additional \href{https://www.openmediavault.org/?page_id=1206}{Volker Theile},
the founder of the \gls{OMV} project, and his team have done a very good job.
Thank you guys!

\tsImage{./images/OMV-Logon.png}{The OMV login page}{width=\textwidth}

\subsection{The disks}

All attached disks are working. The system disk is a NVME SSD with a 250 GB
capacity. All others are 4 TB WD RED disks.

\tsImage{./images/OMV-Disks.png}{The OMV disks}{width=\textwidth}

\subsection{The RAID storage}

Configuring the \href{https://en.wikipedia.org/wiki/RAID}{RAID} was also less
problematic. But there is an important point: As long as the RAID system is
created, do not use the RAID! During the RAID build I was playing around with
\href{https://www.openmediavault.org/?page_id=2014}{OMV plugins} to see the
diskstats -- and damaged the RAID build. After a reboot the system hangs on
the filesystem check. I was shocked - also about the large amount of blocks
the fsck found and tried to repair. It took a while to understand what happens.
Then I've startet from scratch and everything was okay. The creation of the
file system at the end went without problems. A filesystem check found nothing
to do.

\tsImage{./images/OMV-Raid.png}{The OMV RAID}{width=\textwidth}

\subsection{The file system}

This was a simple step -- just create a filesystem on the previously created
RAID volume.

\tsImage{./images/OMV-Filesystem.png}{The OMV filesystem on the RAID}{width=\textwidth}

\subsection{Shared folders}\label{subsec:Shared folders}

Some common folders should be defined before anything else is done:

\begin{itemize}
    \item The folder \tsFontCode{docker} for the use of \gls{Docker}.
    \item The folder \tsFontCode{homes} as base folder for the users home directories.
    \item The folder \tsFontCode{music} for the miniDLNA plugin.
    \item The folder \tsFontCode{quarantine} for the use of \nameref{subsec:ClamAV}.
    \item The folder \tsFontCode{video} also for the miniDLNA plugin.
\end{itemize}

\tsImage{./images/OMV-Sharedfolders.png}{The OMV shared folders}{width=\textwidth}

This \tsTextQuoteG{shared folders} are defined as container to be used inside
of \gls{OMV}. To make them accessible from the network you have to enable
services for them.

\subsection{Users}

In the settings the option \tsFontCode{User home directory} is enabled and
points to the previously created \tsFontCode(homes) folder. Then I created
the users.

\tsImage{./images/OMV-Users-Directory.png}{The OMV users home directory}{width=\textwidth}

\subsection{CIFS shares}

Simple - enable the \tsFontCode{SMB/CIFS} service, enable the \tsFontCode{home
directories} and then create shares for the previously defined \tsFontCode{shared
folders}. Allow read/write access for the administrator of the shares (me),
the others got read access to them.

\tsImage{./images/OMV-CIFS-Settings.png}{The OMV CIFS settings}{width=\textwidth}

\tsImage{./images/OMV-CIFS-Shares.png}{The OMV CIFS shares}{width=\textwidth}

As you can see there is actualy no public access to this shares. This means
that only priviledged users can access them.

\subsection{Tuning}

\subsubsection{SMB/CIFS}

Transfering data from the old \gls{NAS} to the new one takes a lot of time. To
speed up the process I searched for some tuning and found this
\href{https://techie-show.com/open-media-vault-smb-performance-quick-win/}{here}:

\tsImage{./images/OMV-CIFS-Settings-Advanced.png}{The OMV advanced CIFS settings}{width=\textwidth}

\section{OMV plugins}

\subsection{ClamAV}\label{subsec:ClamAV}

To protect the data I want to use an antivirus program. As open source solution
there is \href{https://www.clamav.net}{ClamAV} available -- and also as plugin
for \gls{OMV}.

\tsImage{./images/OMV-Plugin-ClamAV.png}{The OMV ClamAV plugin}{width=\textwidth}

In the setup we use the previously defined quarantine folder.

\tsImage{./images/OMV-Antivirus-Settings.png}{The antivirus settings}{width=\textwidth}

I enabled a scan on access for specific folders.

\tsImage{./images/OMV-Antivirus-OnAccessScan.png}{The antivirus on access scan}{width=\textwidth}

Also I've enabled a scheduled scan for these folders, too.

\tsImage{./images/OMV-Antivirus-ScheduledScan.png}{The antivirus scheduled scan}{width=\textwidth}

Maybe somebody will claim that all scheduled scans run on the same time.\linebreak
Yes -- this was setup to check the power of the CPU. The system load increases
to a load of about 3 -- that's fantastic from my point of view! This means that
there are more than enough reserves. When I'm spreading the schedule I can
lower the load.

\tsImage{./images/OMV-Antivirus-Load.png}{The antivirus system load on scan}{width=\textwidth}

\subsection{MiniDLNA}

To stream music and videos from the \gls{NAS} to the network I'm using the
\href{https://sourceforge.net/projects/minidlna/}{ReadyDLNA} media server
software. In \gls{OMV} this is used with the oldname \tsFontCode{MiniDLNA}.

\tsImage{./images/OMV-Plugin-MiniDLNA.png}{The MiniDLNA plugin}{width=\textwidth}

The basic setup is simple - I've checked the \tsFontCode{Enable} box and that's
it.

\tsImage{./images/OMV-MiniDLNA-Settings.png}{The MiniDLNA settings}{width=\textwidth}

Then I have to define the shares and the kind of content of the shares.

\tsImage{./images/OMV-MiniDLNA-Shares.png}{The MiniDLNA shares}{width=\textwidth}

\subsection{OMV extras}

The \gls{OMVE} are not available in the plugin list. The way to go to install
them is described \href{https://forum.openmediavault.org/index.php?thread/39594-omvextras-for-omv6/}{here}.
Login as user \tsFontCode{root} using SSH and enter the following command:

\begin{figure}[H]
    \centering
    \begin{tiny}
        \begin{BVerbatim}
sudo su -
wget -O - https://github.com/OpenMediaVault-Plugin-Developers/packages/raw/master/install | bash
        \end{BVerbatim}
    \end{tiny}
    \tsCaptionLabelFigure{The OMV extras installation}
\end{figure}

Using this plugin enables the \gls{Docker} and \gls{Portainer} installation
to enhance the capabilities of the \gls{NAS}.

\tsImage{./images/OMV-Plugin-OMV-Extras.png}{The OMV extras plugin}{width=\textwidth}

\subsection{Autoshutdown}

To save some energy we want the system to shutdown when not used and to wake up
when used. So we want to use the autoshutdown plugin and the \gls{WOL} feature.

\tsImage{./images/OMV-Plugin-Autoshutdown.png}{The OMV autoshutdown plugin}{width=\textwidth}

As first step we have to enable the \gls{WOL} on the network card. You can find
the checkbox as last option ot the \tsFontCode{Advanced settings} section.

\tsImage{./images/OMV-NIC-WOL.png}{The OMV network card setup}{width=\textwidth}

Then we have to configure/setup the autoshutdown plugin. First step is to
enable the autoshutdown. Second step is to force an uptime between 23:30 and
02:00 to ensure backup is done. The rest is left with default settings and
might be updated at a later point.

\tsImage{./images/OMV-Autoshutdown-Setup-1.png}{The autoshutdown setup 1}{width=\textwidth}

\tsImage{./images/OMV-Autoshutdown-Setup-2.png}{The autoshutdown setup 2}{width=\textwidth}

\tsImage{./images/OMV-Autoshutdown-Setup-3.png}{The autoshutdown setup 3}{width=\textwidth}

\section{Docker}

By using \gls{Docker} I want to enhance the \gls{NAS} with features which are not
available out-of-the-box. Especially with services my previous \gls{NAS} offers
and which are not native available for \gls{OMV}.

\subsection{Docker installation}

The default location of \gls{Docker} is \tsFontCode{/var/lib/docker} on the
system disk. Allthough the system disk is a fast SSD, I want to install
\gls{Docker} on the slower RAID storage. So I have to change the path of
\gls{Docker} storage.

\tsImage{./images/OMV-Docker-Setup.png}{The Docker setup}{width=\textwidth}

In case \gls{Docker} is already installed, see \href{https://www.reddit.com/r/OpenMediaVault/comments/fl40gf/moving_docker_storage/}{here}
how to move the \gls{Docker} storage to another location.

\subsection{Portainer}

To have more comfort in dealing with \gls{Docker} we install also \gls{Portainer}
from the \gls{OMVE}.

\tsImage{./images/OMV-Portainer-Setup.png}{The Portainer setup}{width=\textwidth}

After installation \gls{Portainer} is up and running.

\tsImage{./images/OMV-Portainer-Login.png}{The Portainer login}{width=\textwidth}

\subsection{SCM-Manager}

\gls{SCM} provides a comfortable user interface for \gls{git},
\href{https://www.mercurial-scm.org}{Mercurial} and \gls{svn}. Up to now I use
\gls{svn} for \href{https://en.wikipedia.org/wiki/Version_control}{version control}.
Some of my repositories are private and I will never publish them to a public
hoster like \href{https://github.com}{GitHub} allthough they offer private
repositories. All my other repositories are hosted on my \gls{NAS} -- the plan
is to move them to \gls{git} if possible -- just to have them on a more modern
version control system.
\bigbreak

First of all we have to create a \gls{Docker} volume for \gls{SCM}. This is
important for running the backup script.

\tsImage{./images/Docker-Volume-SCM.png}{The docker volume scmmanager}{width=\textwidth}

Then we can setup the container:

\begin{itemize}
    \item Mapping port \tsFontCode{2222} to \tsFontCode{2222} to enable the \gls{svn} protocol
    \item Mapping port \tsFontCode{8080} to \tsFontCode{8080} to enable the http protocol
    \item Mapping of the \gls{SCM} volume to \tsFontCode{/var/lib/scm}
\end{itemize}

\tsImage{./images/Docker-Container-SCM-Manager-Setup-1.png}{The docker container scmmanager setup 1}{width=\textwidth}

\tsImage{./images/Docker-Container-SCM-Manager-Setup-2.png}{The docker container scmmanager setup 2}{width=\textwidth}

The startup of the container should work without any problems.

\tsImage{./images/SCM-Login.png}{The SCM-Manager login}{width=\textwidth}

After importing my repositories, I've setup a backup job. For this I've updated
my script \href{https://github.com/ThirtySomething/NAS}{svnExport.sh} that it
will work for \gls{SCM}. Setting up a cronjob inside a \gls{Docker} container
is a \href{https://en.wiktionary.org/wiki/pain_in_the_ass}{pita}. The previously
created volume is required to dynamically export all existing \gls{svn}
repositories. The command will be something like this one:

\begin{figure}[H]
    \scriptsize
    \centering
    \begin{BVerbatim}
/bin/bash /srv/<omv-raid>/<path-to-script>/svnExport.sh >>
/srv/<omv-raid>/<path-to-script>/svnExport.log 2>&1 &
    \end{BVerbatim}
    \tsCaptionLabelFigure{Cron command for svnExport}
\end{figure}

The job is then setup in \gls{OMV} scheduled tasks accessing the script.
\tsFontBold{NOTE:} The path to the \gls{SCM} repositories needs to be set
inside in the script.

\tsImage{./images/SCM-Backup-SVN.png}{The SCM-Manager backup task}{width=\textwidth}
