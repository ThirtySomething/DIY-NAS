%% -----------------------------------------------------------------------------
%% MIT License
%%
%% Copyright 2022 ThirtySomething
%%
%% Permission is hereby granted, free of charge, to any person obtaining a
%% copy of this software and associated documentation files (the "Software"),
%% to deal in the Software without restriction, including without limitation
%% the rights to use, copy, modify, merge, publish, distribute, sublicense,
%% and/or sell copies of the Software, and to permit persons to whom the
%% Software is furnished to do so, subject to the following conditions:
%%
%% The above copyright notice and this permission notice shall be included
%% in all copies or substantial portions of the Software.
%%
%% THE SOFTWARE IS PROVIDED "AS IS", WITHOUT WARRANTY OF ANY KIND, EXPRESS
%% OR IMPLIED, INCLUDING BUT NOT LIMITED TO THE WARRANTIES OF MERCHANTABILITY,
%% FITNESS FOR A PARTICULAR PURPOSE AND NONINFRINGEMENT. IN NO EVENT SHALL
%% THE AUTHORS OR COPYRIGHT HOLDERS BE LIABLE FOR ANY CLAIM, DAMAGES OR OTHER
%% LIABILITY, WHETHER IN AN ACTION OF CONTRACT, TORT OR OTHERWISE, ARISING
%% FROM, OUT OF OR IN CONNECTION WITH THE SOFTWARE OR THE USE OR OTHER
%% DEALINGS IN THE SOFTWARE.
%% -----------------------------------------------------------------------------

%% -----------------------------------------------------------------------------
%% Content of DIY NAS description
%% -----------------------------------------------------------------------------

\subsection{SCM-Manager}

\gls{SCM} provides a comfortable user interface for \gls{git},
\href{https://www.mercurial-scm.org}{Mercurial} and \gls{svn}. Up to now I use
\gls{svn} for \href{https://en.wikipedia.org/wiki/Version_control}{version control}.
Some of my repositories are private and I will never publish them to a public
hoster like \href{https://github.com}{GitHub} allthough they offer private
repositories. All my other repositories are hosted on my \gls{NAS} -- the plan
is to move them to \gls{git} if possible -- just to have them on a more modern
version control system.
\bigbreak

The definition for \gls{DoCo} looks like:

\begin{figure}[H]
    \scriptsize
    \centering
    \begin{BVerbatim}
version: '3.7'
services:
  scmmanager:
    image: cloudogu/scm-manager:latest
    restart: unless-stopped
    ports:
      - 2222:2222
      - 8080:8080
    volumes:
      - scm-config:/var/cache/scm/work
      - scm-data:/var/lib/scm
volumes:
  scm-config:
  scm-data:
    \end{BVerbatim}
    \tsCaptionLabelFigure{docker-compose.yaml for SCM-Manager}
\end{figure}

The startup of the container should work without any problems.

\tsImageO{./images/SCM-Login.png}{The SCM-Manager login}{\custCopyright{}}{width=\textwidth}

After importing my repositories, I've setup a backup job. For this I've updated
my script \href{https://github.com/ThirtySomething/NAS}{svnExport.sh} that it
will work for \gls{SCM}. Setting up a cronjob inside a \gls{Docker} container
is a \href{https://en.wiktionary.org/wiki/pain_in_the_ass}{pita}. The previously
created volume is required to dynamically export all existing \gls{svn}
repositories. The command will be something like this one:

\begin{figure}[H]
    \scriptsize
    \centering
    \begin{BVerbatim}
/bin/bash /srv/<omv-raid>/<path-to-script>/svnExport.sh >>
/srv/<omv-raid>/<path-to-script>/svnExport.log 2>&1 &
    \end{BVerbatim}
    \tsCaptionLabelFigure{Cron command for svnExport}
\end{figure}

For the script please see appendix \ref{appendix:svnExport.sh}.
\bigbreak

The job is then setup in \gls{OMV} scheduled tasks accessing the script.
\tsTextBold{NOTE:} The path to the \gls{SCM} repositories needs to be set
inside in the script.

\tsImageO{./images/SCM-Backup-SVN.png}{The SCM-Manager backup task}{\custCopyright{}}{width=\textwidth}
