%% -----------------------------------------------------------------------------
%% MIT License
%%
%% Copyright 2022 ThirtySomething
%%
%% Permission is hereby granted, free of charge, to any person obtaining a
%% copy of this software and associated documentation files (the "Software"),
%% to deal in the Software without restriction, including without limitation
%% the rights to use, copy, modify, merge, publish, distribute, sublicense,
%% and/or sell copies of the Software, and to permit persons to whom the
%% Software is furnished to do so, subject to the following conditions:
%%
%% The above copyright notice and this permission notice shall be included
%% in all copies or substantial portions of the Software.
%%
%% THE SOFTWARE IS PROVIDED "AS IS", WITHOUT WARRANTY OF ANY KIND, EXPRESS
%% OR IMPLIED, INCLUDING BUT NOT LIMITED TO THE WARRANTIES OF MERCHANTABILITY,
%% FITNESS FOR A PARTICULAR PURPOSE AND NONINFRINGEMENT. IN NO EVENT SHALL
%% THE AUTHORS OR COPYRIGHT HOLDERS BE LIABLE FOR ANY CLAIM, DAMAGES OR OTHER
%% LIABILITY, WHETHER IN AN ACTION OF CONTRACT, TORT OR OTHERWISE, ARISING
%% FROM, OUT OF OR IN CONNECTION WITH THE SOFTWARE OR THE USE OR OTHER
%% DEALINGS IN THE SOFTWARE.
%% -----------------------------------------------------------------------------

%% -----------------------------------------------------------------------------
%% Content of DIY NAS description
%% -----------------------------------------------------------------------------

\subsection{Syncthing}

\gls{ST} is a file synchronization service. I use this to backup automatically
data from my smartphone to the PC and then from the PC to the NAS. The setup
is easy regarding the \href{https://github.com/syncthing/syncthing/blob/main/README-Docker.md}{Docker README}
of the project.
\bigbreak

First of all I've created a \gls{Docker} volume. In the next step I've setup
the container regarding the ports and map the created volume to \tsTextMonospace{/var/syncthing}.
An important step is to set the environment variables \tsTextMonospace{PUID} and
\tsTextMonospace{GUID}. Then the container will run \tsTextMonospace{unless stopped}.
\bigbreak

Setting up the synchronized folders is as usual with \gls{ST}. The result
may look like this one.

\tsImage{./images/Syncthing.png}{Syncthing}{width=\textwidth}
