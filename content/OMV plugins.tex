%% -----------------------------------------------------------------------------
%% MIT License
%%
%% Copyright 2022 ThirtySomething
%%
%% Permission is hereby granted, free of charge, to any person obtaining a
%% copy of this software and associated documentation files (the "Software"),
%% to deal in the Software without restriction, including without limitation
%% the rights to use, copy, modify, merge, publish, distribute, sublicense,
%% and/or sell copies of the Software, and to permit persons to whom the
%% Software is furnished to do so, subject to the following conditions:
%%
%% The above copyright notice and this permission notice shall be included
%% in all copies or substantial portions of the Software.
%%
%% THE SOFTWARE IS PROVIDED "AS IS", WITHOUT WARRANTY OF ANY KIND, EXPRESS
%% OR IMPLIED, INCLUDING BUT NOT LIMITED TO THE WARRANTIES OF MERCHANTABILITY,
%% FITNESS FOR A PARTICULAR PURPOSE AND NONINFRINGEMENT. IN NO EVENT SHALL
%% THE AUTHORS OR COPYRIGHT HOLDERS BE LIABLE FOR ANY CLAIM, DAMAGES OR OTHER
%% LIABILITY, WHETHER IN AN ACTION OF CONTRACT, TORT OR OTHERWISE, ARISING
%% FROM, OUT OF OR IN CONNECTION WITH THE SOFTWARE OR THE USE OR OTHER
%% DEALINGS IN THE SOFTWARE.
%% -----------------------------------------------------------------------------

%% -----------------------------------------------------------------------------
%% Content of DIY NAS description
%% -----------------------------------------------------------------------------

\section{OMV plugins}

\subsection{ClamAV}\label{subsec:ClamAV}

To protect the data I want to use an antivirus program. As open source solution
there is \href{https://www.clamav.net}{ClamAV} available -- and also as plugin
for \gls{OMV}.

\tsImage{./images/OMV-Plugin-ClamAV.png}{The OMV ClamAV plugin}{width=\textwidth}

In the setup we use the previously defined quarantine folder.

\tsImage{./images/OMV-Antivirus-Settings.png}{The antivirus settings}{width=\textwidth}

I enabled a scan on access for specific folders.

\tsImage{./images/OMV-Antivirus-OnAccessScan.png}{The antivirus on access scan}{width=\textwidth}

Also I've enabled a scheduled scan for these folders, too.

\tsImage{./images/OMV-Antivirus-ScheduledScan.png}{The antivirus scheduled scan}{width=\textwidth}

Maybe somebody will claim that all scheduled scans run on the same time.\linebreak
Yes -- this was setup to check the power of the CPU. The system load increases
to a load of about 3 -- that's fantastic from my point of view! This means that
there are more than enough reserves. When I'm spreading the schedule I can
lower the load.

\tsImage{./images/OMV-Antivirus-Load.png}{The antivirus system load on scan}{width=\textwidth}

\subsection{MiniDLNA}

To stream music and videos from the \gls{NAS} to the network I'm using the
\href{https://sourceforge.net/projects/minidlna/}{ReadyDLNA} media server
software. In \gls{OMV} this is used with the oldname \tsFontCode{MiniDLNA}.

\tsImage{./images/OMV-Plugin-MiniDLNA.png}{The MiniDLNA plugin}{width=\textwidth}

The basic setup is simple - I've checked the \tsFontCode{Enable} box and that's
it.

\tsImage{./images/OMV-MiniDLNA-Settings.png}{The MiniDLNA settings}{width=\textwidth}

Then I have to define the shares and the kind of content of the shares.

\tsImage{./images/OMV-MiniDLNA-Shares.png}{The MiniDLNA shares}{width=\textwidth}

\subsection{OMV extras}

The \gls{OMVE} are not available in the plugin list. The way to go to install
them is described \href{https://forum.openmediavault.org/index.php?thread/39594-omvextras-for-omv6/}{here}.
Login as user \tsFontCode{root} using SSH and enter the following command:

\begin{figure}[H]
    \centering
    \begin{tiny}
        \begin{BVerbatim}
sudo su -
wget -O - https://github.com/OpenMediaVault-Plugin-Developers/packages/raw/master/install | bash
        \end{BVerbatim}
    \end{tiny}
    \tsCaptionLabelFigure{The OMV extras installation}
\end{figure}

Using this plugin enables the \gls{Docker} and \gls{Portainer} installation
to enhance the capabilities of the \gls{NAS}.

\tsImage{./images/OMV-Plugin-OMV-Extras.png}{The OMV extras plugin}{width=\textwidth}

\subsection{Autoshutdown}

To save some energy we want the system to shutdown when not used and to wake up
when used. So we want to use the autoshutdown plugin and the \gls{WOL} feature.

\tsImage{./images/OMV-Plugin-Autoshutdown.png}{The OMV autoshutdown plugin}{width=\textwidth}

As first step we have to enable the \gls{WOL} on the network card. You can find
the checkbox as last option ot the \tsFontCode{Advanced settings} section.

\tsImage{./images/OMV-NIC-WOL.png}{The OMV network card setup}{width=\textwidth}

Then we have to configure/setup the autoshutdown plugin. First step is to
enable the autoshutdown. Second step is to force an uptime between 23:30 and
02:00 to ensure backup is done. The rest is left with default settings and
might be updated at a later point.

\tsImage{./images/OMV-Autoshutdown-Setup-1.png}{The autoshutdown setup 1}{width=\textwidth}

\tsImage{./images/OMV-Autoshutdown-Setup-2.png}{The autoshutdown setup 2}{width=\textwidth}

\tsImage{./images/OMV-Autoshutdown-Setup-3.png}{The autoshutdown setup 3}{width=\textwidth}
