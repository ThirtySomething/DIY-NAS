%% -----------------------------------------------------------------------------
%% MIT License
%%
%% Copyright 2022 ThirtySomething
%%
%% Permission is hereby granted, free of charge, to any person obtaining a
%% copy of this software and associated documentation files (the "Software"),
%% to deal in the Software without restriction, including without limitation
%% the rights to use, copy, modify, merge, publish, distribute, sublicense,
%% and/or sell copies of the Software, and to permit persons to whom the
%% Software is furnished to do so, subject to the following conditions:
%%
%% The above copyright notice and this permission notice shall be included
%% in all copies or substantial portions of the Software.
%%
%% THE SOFTWARE IS PROVIDED "AS IS", WITHOUT WARRANTY OF ANY KIND, EXPRESS
%% OR IMPLIED, INCLUDING BUT NOT LIMITED TO THE WARRANTIES OF MERCHANTABILITY,
%% FITNESS FOR A PARTICULAR PURPOSE AND NONINFRINGEMENT. IN NO EVENT SHALL
%% THE AUTHORS OR COPYRIGHT HOLDERS BE LIABLE FOR ANY CLAIM, DAMAGES OR OTHER
%% LIABILITY, WHETHER IN AN ACTION OF CONTRACT, TORT OR OTHERWISE, ARISING
%% FROM, OUT OF OR IN CONNECTION WITH THE SOFTWARE OR THE USE OR OTHER
%% DEALINGS IN THE SOFTWARE.
%% -----------------------------------------------------------------------------

%% -----------------------------------------------------------------------------
%% Content of DIY NAS description
%% -----------------------------------------------------------------------------

\section{Docker}

By using \gls{doc} I want to enhance the \gls{nas} with features which are not
available out-of-the-box. Especially with services my previous \gls{nas} offers
and which are not native available for \gls{omv}.

\subsection{Docker installation}

The default location of \gls{doc} is \tsTextMonospace{/var/lib/docker} on the
system disk. Allthough the system disk is a fast SSD, I want to install
\gls{doc} on the slower RAID storage. So I have to change the path of
\gls{doc} storage.

\tsImageO{./images/OMV-Docker-Setup.png}{The Docker setup}{\custCopyright{}}{width=\textwidth}

In case \gls{doc} is already installed, see \href{https://www.reddit.com/r/OpenMediaVault/comments/fl40gf/moving_docker_storage/}{here}
how to move the \gls{doc} storage to another location.

\subsection{Portainer}

To have more comfort in dealing with \gls{doc} we install also \gls{pot}
from the \gls{ove}.

\tsImageO{./images/OMV-Portainer-Setup.png}{The Portainer setup}{\custCopyright{}}{width=\textwidth}

After installation \gls{pot} is up and running.

\tsImageO{./images/OMV-Portainer-Login.png}{The Portainer login}{\custCopyright{}}{width=\textwidth}

\subsection{Container list}

\begin{figure}[H]
    \begin{itemize}
        \item \tsTextStrikethrough{\gls{pot} -- UI for docker}
              -- done with docker installation.
        \item \tsTextStrikethrough{\gls{scm} -- SCM with git, svn and hg}
              -- done.
        \item \gls{mdb} -- Database for various projects
        \item \gls{pma} -- UI for the database
        \item \gls{mos} -- A \href{https://mqtt.org/}{MQTT} broker
        \item \tsTextStrikethrough{\gls{syt} -- File synchronization}
        \item \gls{ncl} -- Private cloud service
        \item Backup server?
              \begin{itemize}
                  \item \gls{rtc}
                  \item \gls{dup}
              \end{itemize}
        \item \gls{mio} -- As data sink for \gls{ncl},
              \gls{rtc} or \gls{dup}
        \item \tsTextStrikethrough{\href{https://pi-hole.net/}{Pi-hole} -- Adblocker for the network}
              -- won't do that, see section \ref{subsec:Pi-hole} for explanation
        \item \href{https://gotify.net/}{gotify} -- Notification server
    \end{itemize}
    \tsCaptionLabelFigure{Containerlist}
\end{figure}

\custInputFile{DockerContainer}
