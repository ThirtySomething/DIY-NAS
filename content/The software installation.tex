%% -----------------------------------------------------------------------------
%% MIT License
%%
%% Copyright 2022 ThirtySomething
%%
%% Permission is hereby granted, free of charge, to any person obtaining a
%% copy of this software and associated documentation files (the "Software"),
%% to deal in the Software without restriction, including without limitation
%% the rights to use, copy, modify, merge, publish, distribute, sublicense,
%% and/or sell copies of the Software, and to permit persons to whom the
%% Software is furnished to do so, subject to the following conditions:
%%
%% The above copyright notice and this permission notice shall be included
%% in all copies or substantial portions of the Software.
%%
%% THE SOFTWARE IS PROVIDED "AS IS", WITHOUT WARRANTY OF ANY KIND, EXPRESS
%% OR IMPLIED, INCLUDING BUT NOT LIMITED TO THE WARRANTIES OF MERCHANTABILITY,
%% FITNESS FOR A PARTICULAR PURPOSE AND NONINFRINGEMENT. IN NO EVENT SHALL
%% THE AUTHORS OR COPYRIGHT HOLDERS BE LIABLE FOR ANY CLAIM, DAMAGES OR OTHER
%% LIABILITY, WHETHER IN AN ACTION OF CONTRACT, TORT OR OTHERWISE, ARISING
%% FROM, OUT OF OR IN CONNECTION WITH THE SOFTWARE OR THE USE OR OTHER
%% DEALINGS IN THE SOFTWARE.
%% -----------------------------------------------------------------------------

%% -----------------------------------------------------------------------------
%% Content of DIY NAS description
%% -----------------------------------------------------------------------------

\section{The software installation}

\subsection{The operating system}

The basic installation of \gls{omv} works without any problems. This is known
from debian. Additional \href{https://www.openmediavault.org/?page_id=1206}{Volker Theile},
the founder of the \gls{omv} project, and his team have done a very good job.
Thank you guys!

\tsImageO{./images/OMV-Logon.png}{The OMV login page}{\custCopyright{}}{width=\textwidth}

\subsection{The disks}

All attached disks are working. The system disk is a NVME SSD with a 250 GB
capacity. All others are 4 TB WD RED disks.

\tsImageO{./images/OMV-Disks.png}{The OMV disks}{\custCopyright{}}{width=\textwidth}

\subsection{The RAID storage}

Configuring the \href{https://en.wikipedia.org/wiki/RAID}{RAID} was also less
problematic. But there is an important point: As long as the RAID system is
created, do not use the RAID! During the RAID build I was playing around with
\href{https://www.openmediavault.org/?page_id=2014}{OMV plugins} to see the
diskstats -- and damaged the RAID build. After a reboot the system hangs on
the filesystem check. I was shocked - also about the large amount of blocks
the fsck found and tried to repair. It took a while to understand what happens.
Then I've startet from scratch and everything was okay. The creation of the
file system at the end went without problems. A filesystem check found nothing
to do.

\tsImageO{./images/OMV-Raid.png}{The OMV RAID}{\custCopyright{}}{width=\textwidth}

\subsection{The file system}

This was a simple step -- just create a filesystem on the previously created
RAID volume.

\tsImageO{./images/OMV-Filesystem.png}{The OMV filesystem on the RAID}{\custCopyright{}}{width=\textwidth}

\subsection{Shared folders}\label{subsec:Shared folders}

Some common folders should be defined before anything else is done:

\begin{itemize}
    \item The folder \tsTextMonospace{docker} for the use of \gls{doc}.
    \item The folder \tsTextMonospace{homes} as base folder for the users home directories.
    \item The folder \tsTextMonospace{music} for the miniDLNA plugin.
    \item The folder \tsTextMonospace{quarantine} for the use of \nameref{subsec:ClamAV}.
    \item The folder \tsTextMonospace{video} also for the miniDLNA plugin.
\end{itemize}

\tsImageO{./images/OMV-Sharedfolders.png}{The OMV shared folders}{\custCopyright{}}{width=\textwidth}

This \tsTextQuoteG{shared folders} are defined as container to be used inside
of \gls{omv}. To make them accessible from the network you have to enable
services for them.

\subsection{Users}

In the settings the option \tsTextMonospace{User home directory} is enabled and
points to the previously created \tsTextMonospace{homes} folder. Then I created
the users.

\tsImageO{./images/OMV-Users-Directory.png}{The OMV users home directory}{\custCopyright{}}{width=\textwidth}

\subsection{CIFS shares}

Simple - enable the \tsTextMonospace{SMB/CIFS} service, enable the \tsTextMonospace{home
directories} and then create shares for the previously defined \tsTextMonospace{shared
folders}. Allow read/write access for the administrator of the shares (me),
the others got read access to them.

\tsImageO{./images/OMV-CIFS-Settings.png}{The OMV CIFS settings}{\custCopyright{}}{width=\textwidth}

\tsImageO{./images/OMV-CIFS-Shares.png}{The OMV CIFS shares}{\custCopyright{}}{width=\textwidth}

As you can see there is actualy no public access to this shares. This means
that only priviledged users can access them.

\subsection{Tuning}

\subsubsection{SMB/CIFS}

Transfering data from the old \gls{nas} to the new one takes a lot of time. To
speed up the process I searched for some tuning and found this
\href{https://techie-show.com/open-media-vault-smb-performance-quick-win/}{here}:

\tsImageO{./images/OMV-CIFS-Settings-Advanced.png}{The OMV advanced CIFS settings}{\custCopyright{}}{width=\textwidth}
